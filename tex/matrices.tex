\appendix
\section*{Addendum: Automata as matrix multiplication}%
\label{sec:matrix}

We probably won't have time to cover this in class, but I figured it's too cool a thing not to include on the handout.

\subsection*{Background: Matrix multiplication}
\label{sub:matrix_multiplication}

You all probably know how to calculate the dot product of two vectors.
First you multiply their components to get a single vector, and then you calculate its magnitude by summing.
For example:

\[
      \begin{pmatrix}0\\ 2\\ 3\end{pmatrix} \cdot \begin{pmatrix}8\\5\\2\end{pmatrix}
    = \left | \begin{pmatrix}0 \cdot 8\\ 2 \cdot 5\\ 3 \cdot 2\end{pmatrix} \right |
    = \left | \begin{pmatrix}0\\10\\6\end{pmatrix} \right |
    = 0 + 10 + 6\\
    = 16
\]

Matrices are a generalization of vectors where we can have multiple columns in addition to multiple rows.

\[
    \begin{pmatrix}
        5 & 2\\
        3 & 1\\
        0 & 2\\
    \end{pmatrix}
\]

Like vectors, matrices can also be multiplied, but things are slightly different.
First, the number of columns of the first matrix must match the number of rows of the second matrix.

\[
    \begin{pmatrix}
        5 & 2\\
        3 & 1\\
        0 & 2\\
    \end{pmatrix}
    \cdot
    \begin{pmatrix}
        1 & 2 & 0 & 2\\
        3 & 0 & 1 & 3\\
    \end{pmatrix}
    =
    \text{\textbf{\color{blue!75}alright, let's do this!}}
\]

\[
    \begin{pmatrix}
        5 & 2\\
        3 & 1\\
        0 & 2\\
    \end{pmatrix}
    \cdot
    \begin{pmatrix}
        1 & 2 & 1\\
        3 & 0 & 2\\
        5 & 1 & 0\\
    \end{pmatrix}
    =
    \text{\textbf{\color{red!75}sorry, no can do!}}
\]

As long as that is the case, we can multiply two matrices as follows:

\begin{enumerate}
    \item Write down the \textbf{first row of the first matrix} as a vector (top-down rather than left-to-right).
    \item Calculate the dot product of this vector and the \textbf{first column of the second matrix}.
          Write it down.
    \item Do the same with the \textbf{second column of the second matrix}, and put the result next to the one you already wrote down.
    \item Do the same with all other rows of the second matrix.
    \item First cycle done.
          Now repeat the same steps with the \textbf{second row of the first matrix}.
    \item Continue until all rows are done.
          You've got yourself a shiny new matrix.
\end{enumerate}

\begin{examplebox}[Matrix multiplication]
    Let's try the following matrix multiplication:
    \[
        \begin{pmatrix}
            5 & 2\\
            3 & 1\\
            0 & 2\\
        \end{pmatrix}
        \cdot
        \begin{pmatrix}
            1 & 2 & 0 & 2\\
            3 & 0 & 1 & 3\\
        \end{pmatrix}
        =
        \text{\textbf{???}}
    \]
    We write down the first row of the first matrix as a vector.
    \[
        \begin{pmatrix}
            5\\
            2\\
        \end{pmatrix}
    \]
    Next we have to calculate the dot products for each column of the second matrix.
    %
    \[
        \begin{pmatrix}
            5\\
            2\\
        \end{pmatrix}
        \cdot
        \begin{pmatrix}
            1\\
            3\\
        \end{pmatrix}
        =
        11
        \qquad
        \begin{pmatrix}
            5\\
            2\\
        \end{pmatrix}
        \cdot
        \begin{pmatrix}
            2\\
            0\\
        \end{pmatrix}
        =
        10
        \qquad
        \begin{pmatrix}
            5\\
            2\\
        \end{pmatrix}
        \cdot
        \begin{pmatrix}
            0\\
            1\\
        \end{pmatrix}
        =
        2
        \qquad
        \begin{pmatrix}
            5\\
            2\\
        \end{pmatrix}
        \cdot
        \begin{pmatrix}
            2\\
            3\\
        \end{pmatrix}
        =
        16
    \]
    So the first row for the new matrix is $(11\ 10\ 2\ 16)$.
    Repeating the same procedure with the second and third row, we can solve the full equation.
    \[
        \begin{pmatrix}
            5 & 2\\
            3 & 1\\
            0 & 2\\
        \end{pmatrix}
        \cdot
        \begin{pmatrix}
            1 & 2 & 0 & 2\\
            3 & 0 & 1 & 3\\
        \end{pmatrix}
        =
        \begin{pmatrix}
            11 & 10 & 2 & 16\\
            6 & 6 & 1 & 9\\
            6 & 0 & 2 & 6\\
        \end{pmatrix}
    \]
\end{examplebox}

\subsection*{Boolean matrix multiplication for FSAs}%
\label{sub:matrix_boolean}

\emph{Boolean} matrices are a special case where all matrices can only have $0$ and $1$ as their components.
So \emph{Boolean} is just another term for \emph{binary}.
Boolean matrix multiplication works almost exactly like normal matrix multiplication, except that the value of a dot product can never exceed $1$:
\[
    \begin{pmatrix}
        0\\1\\1
    \end{pmatrix}
    \cdot
    \begin{pmatrix}
        1\\1\\1
    \end{pmatrix}
    =
    1
    \text{\quad (not 2)}
\]

Every FSA can be represented as a collection of matrices.
This is best explained with an example.

Here's an automaton we've seen before.
%
\begin{center}
    \begin{tikzpicture}
        \node[state,initial] (0) {0};
        \node[state] (1) [right=of 0] {1};
        \node[state,accepting] (2) [above=4em of 1] {2};
        \node[state] (3) [right=8em of 1] {3};

        \path[->] (0) edge [bend left=65]  node [above] {John} (1)
                      edge                 node [above] {Mary} (1)
                      edge [bend right=65] node [below] {Bill} (1)
                  (1) edge                 node [right] {left} (2)
                      edge [bend right=20] node [below] {thinks} (3)
                  (3) edge [bend right=60] node [above] {John} (1)
                  (3) edge [bend right=20] node [above] {Mary} (1)
                  (3) edge [bend left=60]  node [below] {Bill} (1);
    \end{tikzpicture}
\end{center}

We already know that it accepts \emph{John thinks Mary left}, but not \emph{John thinks Mary left Bill}.
We can get the very same result through matrix multiplication.

First, every one of the symbols \emph{John}, \emph{Mary}, \emph{Bill}, \emph{thinks}, and \emph{left} is given associated with a matrix that encodes whether one can go from some state to another state with this symbol.
This works as follows.
Imagine that we identify each state with a row and with a column as in the template below.
%
\[
    \begin{array}{cc}
        &
        \begin{array}{cccc}
            \textbf{0} &
            \textbf{1} &
            \textbf{2} &
            \textbf{3} \\
        \end{array}
        \\
        \begin{array}{cccc}
            \textbf{0} \\
            \textbf{1} \\
            \textbf{2} \\
            \textbf{3} \\
        \end{array}
        &
        \left (
            \begin{array}{cccc}
                \phantom{0} & \phantom{1} & \phantom{0} & \phantom{0}\\
                \phantom{0} & \phantom{0} & \phantom{0} & \phantom{0}\\
                \phantom{0} & \phantom{0} & \phantom{0} & \phantom{0}\\
                \phantom{0} & \phantom{1} & \phantom{0} & \phantom{0}\\
            \end{array}
        \right )
    \end{array}
\]
%
We put a 1 in a cell in row $x$ and column $y$ iff it is possible to go from state $x$ to state $y$ with the symbol we're building the matrix for.

Here's how this works for John.
With \emph{John} we cannot go from state 0 to state 0, so we put a $0$ in the very first cell.
But we can go from $0$ to state $1$, so the next cell gets a $1$.
\emph{John} by itself cannot take us from state $0$ to state $2$ or $3$, so the next two cells are also $0$.
\[
    \begin{array}{cc}
        &
        \begin{array}{cccc}
            \textbf{0} &
            \textbf{1} &
            \textbf{2} &
            \textbf{3} \\
        \end{array}
        \\
        \begin{array}{cccc}
            \textbf{0} \\
            \textbf{1} \\
            \textbf{2} \\
            \textbf{3} \\
        \end{array}
        &
        \left (
            \begin{array}{cccc}
                0 & 1 & 0 & 0\\
                \phantom{0} & \phantom{0} & \phantom{0} & \phantom{0}\\
                \phantom{0} & \phantom{0} & \phantom{0} & \phantom{0}\\
                \phantom{0} & \phantom{1} & \phantom{0} & \phantom{0}\\
            \end{array}
        \right )
    \end{array}
\]
We could in fill the other rows step by step, too, but there's a smarter way.
Looking at the automaton, we can see immediately that \emph{John} can only take us from $0$ to $1$ and from $3$ to $1$.
The first one we've already kept track of in the first row, leaving only the move from $3$ to $1$.
To capture this, we just put a $1$ in the second column of the last row.
All other cells are filled with $0$.
\[
    \begin{array}{cc}
        &
        \begin{array}{cccc}
            \textbf{0} &
            \textbf{1} &
            \textbf{2} &
            \textbf{3} \\
        \end{array}
        \\
        \begin{array}{cccc}
            \textbf{0} \\
            \textbf{1} \\
            \textbf{2} \\
            \textbf{3} \\
        \end{array}
        &
        \left (
            \begin{array}{cccc}
                0 & 1 & 0 & 0\\
                0 & 0 & 0 & 0\\
                0 & 0 & 0 & 0\\
                0 & 1 & 0 & 0\\
            \end{array}
        \right )
    \end{array}
\]
Repeating the same procedure for the other symbols, we get the following matrices:
%
\[
    \begin{array}{ccc}
        \textbf{John}/\textbf{Mary}/\textbf{Bill} & \textbf{thinks} & \textbf{left}\\
        \begin{pmatrix}
            0 & 1 & 0 & 0\\
            0 & 0 & 0 & 0\\
            0 & 0 & 0 & 0\\
            0 & 1 & 0 & 0\\
        \end{pmatrix}
        &
        \begin{pmatrix}
            0 & 0 & 0 & 0\\
            0 & 0 & 0 & 1\\
            0 & 0 & 0 & 0\\
            0 & 0 & 0 & 0\\
        \end{pmatrix}
        &
        \begin{pmatrix}
            0 & 0 & 0 & 0\\
            0 & 0 & 1 & 0\\
            0 & 0 & 0 & 0\\
            0 & 0 & 0 & 0\\
        \end{pmatrix}
    \end{array}
\]
%
We're almost done, only two more matrices are needed.
One records which states we can go to from the start.
%
\[
    \begin{array}{cc}
        &
        \begin{array}{cccc}
            \textbf{0} &
            \textbf{1} &
            \textbf{2} &
            \textbf{3} \\
        \end{array}
        \\
        \textbf{start}
        &
        \left (
            \begin{array}{cccc}
                1 & 0 & 0 & 0\\
            \end{array}
        \right )
    \end{array}
\]
%
And the other records from which states we can move to the exit.
\[
    \begin{array}{cc}
        &
        \textbf{exit}
        \\
        \begin{array}{c}
            \textbf{0} \\
            \textbf{1} \\
            \textbf{2} \\
            \textbf{3} \\
        \end{array}
        &
        \left (
            \begin{array}{c}
                0 \\
                0 \\
                1 \\
                0 \\
            \end{array}
        \right )
    \end{array}
\]

Now we're finally able to compute for any given string whether it is accepted by the automaton.
Let's do the calculation for the accepted \emph{John thinks Mary left}.

\begin{align*}
      & \textbf{start} \cdot \textbf{John} \cdot \textbf{thinks} \cdot \textbf{Mary} \cdot \textbf{left} \cdot \textbf{exit}\\
    = & 
        \begin{pmatrix}
            1 & 0 & 0 & 0\\
        \end{pmatrix}
        \cdot
        \begin{pmatrix}
            0 & 1 & 0 & 0\\
            0 & 0 & 0 & 0\\
            0 & 0 & 0 & 0\\
            0 & 1 & 0 & 0\\
        \end{pmatrix}
        \cdot
        \begin{pmatrix}
            0 & 0 & 0 & 0\\
            0 & 0 & 0 & 1\\
            0 & 0 & 0 & 0\\
            0 & 0 & 0 & 0\\
        \end{pmatrix}
        \cdot
        \begin{pmatrix}
            0 & 1 & 0 & 0\\
            0 & 0 & 0 & 0\\
            0 & 0 & 0 & 0\\
            0 & 1 & 0 & 0\\
        \end{pmatrix}
        \cdot
        \begin{pmatrix}
            0 & 0 & 0 & 0\\
            0 & 0 & 1 & 0\\
            0 & 0 & 0 & 0\\
            0 & 0 & 0 & 0\\
        \end{pmatrix}
        \cdot
        \begin{pmatrix}
            0\\
            0\\
            1\\
            0\\
        \end{pmatrix}
        \\
    = & 
        \begin{pmatrix}
            0 & 1 & 0 & 0\\
        \end{pmatrix}
        \cdot
        \begin{pmatrix}
            0 & 0 & 0 & 0\\
            0 & 0 & 0 & 1\\
            0 & 0 & 0 & 0\\
            0 & 0 & 0 & 0\\
        \end{pmatrix}
        \cdot
        \begin{pmatrix}
            0 & 1 & 0 & 0\\
            0 & 0 & 0 & 0\\
            0 & 0 & 0 & 0\\
            0 & 1 & 0 & 0\\
        \end{pmatrix}
        \cdot
        \begin{pmatrix}
            0 & 0 & 0 & 0\\
            0 & 0 & 1 & 0\\
            0 & 0 & 0 & 0\\
            0 & 0 & 0 & 0\\
        \end{pmatrix}
        \cdot
        \begin{pmatrix}
            0\\
            0\\
            1\\
            0\\
        \end{pmatrix}
        \\
    = & 
        \begin{pmatrix}
            0 & 0 & 0 & 1\\
        \end{pmatrix}
        \cdot
        \begin{pmatrix}
            0 & 1 & 0 & 0\\
            0 & 0 & 0 & 0\\
            0 & 0 & 0 & 0\\
            0 & 1 & 0 & 0\\
        \end{pmatrix}
        \cdot
        \begin{pmatrix}
            0 & 0 & 0 & 0\\
            0 & 0 & 1 & 0\\
            0 & 0 & 0 & 0\\
            0 & 0 & 0 & 0\\
        \end{pmatrix}
        \cdot
        \begin{pmatrix}
            0\\
            0\\
            1\\
            0\\
        \end{pmatrix}
        \\
    = & 
        \begin{pmatrix}
            0 & 1 & 0 & 0\\
        \end{pmatrix}
        \cdot
        \begin{pmatrix}
            0 & 0 & 0 & 0\\
            0 & 0 & 1 & 0\\
            0 & 0 & 0 & 0\\
            0 & 0 & 0 & 0\\
        \end{pmatrix}
        \cdot
        \begin{pmatrix}
            0\\
            0\\
            1\\
            0\\
        \end{pmatrix}
        \\
    = & 
        \begin{pmatrix}
            0 & 0 & 1 & 0\\
        \end{pmatrix}
        \cdot
        \begin{pmatrix}
            0\\
            0\\
            1\\
            0\\
        \end{pmatrix}
        \\
    = & 1
\end{align*}
%
We got a $1$.
If you do the calculation with the bad sentence \emph{John thinks Mary}, instead, you'll get a $0$.
So good sentences always result in a $1$, bad sentences in a $0$.

\medskip
\begin{homework}
    Calculate the values for the following strings:
    %
    \begin{enumerate}
        \item John
        \item John thinks Mary left Bill.
        \item John thinks Bill thinks Mary left.
    \end{enumerate}
\end{homework}

\medskip
\begin{homework}
    Consider the automaton below.
    %
    \begin{center}
        \begin{tikzpicture}
            \node[state,initial] (0) {0};
            \node[state,accepting] (2) [right=of 0] {1};
            \node[state,accepting] (3) [right=of 2] {2};
            \node[state,initial] (1) [below=of 3] {3};

            \foreach \Source/\Target/\Label in {%
                0/2/a,
                2/3/d%
                }
                \path[->] (\Source) edge node [above] {$\Label$} (\Target);
            \path[->] (1) edge node [left] {$b$} (3);
            \path[->] (0) edge [loop above] node[above] {$a$} (0);
            \path[->] (2) edge [loop above] node[above] {$c$} (2);
        \end{tikzpicture}
    \end{center}
    %
    Represent it with Boolean matrices and the calculate the values of the following strings:
    %
    \begin{enumerate}
        \item $a$
        \item $b$
        \item $aa$
        \item $ad$
        \item $bd$
        \item $accccc$
    \end{enumerate}
\end{homework}

\noindent
\textbf{Bottom-line:} FSAs $\approx$ specific kind of matrix multiplication
