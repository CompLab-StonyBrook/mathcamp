\chapter*{\texorpdfstring{\color{brown!95}Automata Theory for Language}{Automata Theory for Language}}
\label{cha:fsa}
% \setcounter{page}{1}

\fcolorbox{brown!25}{brown!25}{%
    \centering
    \begin{tabular}{ll}
        \textbf{Name:} Thomas Graf &
        \textbf{Email:} \href{mailto:mathcamp@thomasgraf.net}{mathcamp@thomasgraf.net}\\
        \textbf{Course Website:} \href{http://mathcamp.thomasgraf.net} {mathcamp.thomasgraf.net} &
        \textbf{Personal Website:} \href{http://thomasgraf.net}{thomasgraf.net}
    \end{tabular}
}

\section{Language as a mathematical problem}

Every language follows certain rules of grammar.
I do not mean the usual grammar-nazi malarkey like ``Do not split infinitives! Do not strand prepositions!''.
Rather, there are words and sentences that are correct, and others that have something wrong with them.

\begin{examplebox}
    Consider the English word \emph{denaturalization}.
    It is built up from discrete parts:
    %
    \begin{enumerate}
        \item nature
        \item nature + al = natural
        \item natural + ize = naturalize
        \item de + naturalize = denaturalize
        \item denaturalize + ation = denaturalization
    \end{enumerate}
    %
    No other order of these parts produces a good word of English:
    
    \smallskip
    \noindent
    \emph{denaturizalation}, \emph{ationalizenaturde}, \emph{naturalizedeation}, \ldots
\end{examplebox}
%
\begin{homework}
    How many combinations are mathematically possible?
\end{homework}

% This restriction to just one order is quite surprising if you think about it.
% In sentences, we can arrange things much more freely:
% %
% \begin{examplebox}
%     Even in a simple sentence like \emph{Sue and Mary did not meet all the boys yesterday} we can change the order of words quite a bit:
%     %
%     \begin{enumerate}
%         \item Sue and Mary did not meet all the boys yesterday.
%         \item Mary and Sue did not meet all the boys yesterday.
%         \item All the boys, Mary and Sue did not meet yesterday.
%         \item All the boys, Mary and Sue did not yesterday meet.
%         \item The boys, Mary and Sue did not all meet yesterday.
%         \item Yesterday, Mary and Sue did not meet all the boys.
%         \item \ldots
%     \end{enumerate}
%     %
%     Admittedly, many orders are still impossible, though:
%     %
%     \begin{enumerate}
%         \item Sue not meet yesterday did Mary the all boys.
%         \item Sue Mary and did not meet all the boys yesterday.
%         \item Yesterday the boys, Mary and Sue did not all meet.
%         \item The boys, Mary and Sue did not meet all yesterday.
%     \end{enumerate}
% \end{examplebox}
%
We see, then, that there must be a very strict system of rules that determines the order in which parts of a word may appear.
What can we say about this rule system?
Can we study it just like, say, the laws that determine how atoms combine into molecules?

One strategy would be to look at the human brain.
After all, language must involve some kind of brain mechanisms that allow us to produce correct forms while avoiding incorrect ones.
But research in \emph{neurolinguistics} has shown that this approach won't get us very far --- the brain is just too complicated to be studied this way.
Instead, we need a more abstract model of these brain mechanisms.
And math allows us to do just that!

\section{Graphs for language: Finite-state automata}

Our mathematical model are \emph{finite-state automata} (FSA), which are a special case of labeled graphs.
Here is a simple example of an automaton representing \emph{denaturalization}:
%
\begin{center}
    \begin{tikzpicture}
        \node[state,initial] (0) {0};
        \foreach \New/\Old in {%
            1/0,
            2/1,
            3/2,
            4/3}
            \node[state] (\New) [right=of \Old] {\New};
        \node[state,accepting] (5) [right=of 4] {5};

        \foreach \Source/\Target/\Label in {%
            0/1/de,
            1/2/nature,
            2/3/al,
            3/4/ize,
            4/5/ation}
            \path[->] (\Source) edge node [above] {\Label} (\Target);
    \end{tikzpicture}
\end{center}
%
\begin{itemize}
    \item The circles are the vertices of the graph, which are called \emph{states}.
    \item The \emph{arcs} connecting the states all must have a label.
    \item The \emph{start} arrow indicates the beginning point of the automaton, called the \emph{initial state}.
    \item The doubly circled state is a \emph{final state}.
        It marks the end point of the automaton.
\end{itemize}

An FSA can be used to succinctly describe words and sentences.
Every path through the automaton that takes you from an initial state to a final state represents a well-formed structure.
In the automaton above, there is only one such path, which takes us from 0 to 1, then to 2, 3, 4, and finally 5.
But automata can have multiple paths through them.

\begin{examplebox}
    Besides \emph{denaturalization}, English also has the words
     
    \begin{itemize}
        \item \emph{nature},
        \item \emph{natural},
        \item \emph{naturalize},
        \item \emph{denaturalize},
        \item \emph{naturalization}.
    \end{itemize}
     
    We can modify the previous automaton so that it also allows for these other forms. 
    Let's do this step by step.

    First we want to allow words to also end with \emph{nature}, \emph{al}, and \emph{ize}, so we make the states that are reached through those arcs final.
    %
    \begin{center}
        \begin{tikzpicture}
            \node[state,initial] (0) {0};
            \node[state] [right=of 0] {1};
            \foreach \New/\Old in {%
                2/1,
                3/2,
                4/3,
                5/4}
                \node[state,accepting] (\New) [right=of \Old] {\New};

            \foreach \Source/\Target/\Label in {%
                0/1/de,
                1/2/nature,
                2/3/al,
                3/4/ize,
                4/5/ation}
                \path[->] (\Source) edge node [above] {\Label} (\Target);
        \end{tikzpicture}
    \end{center}
    
    But every path must still start with \emph{de}, so the automaton does not allow the patterns \emph{nature}, \emph{natural}, or \emph{naturalize}.
    To fix this, we make 1 an initial state, too.
    %
    \begin{center}
        \begin{tikzpicture}
            \node[state,initial] (0) {0};
            \node[state,initial] (1) [below=of 0] {1};
            \foreach \New/\Old in {%
                2/1,
                3/2,
                4/3,
                5/4}
                \node[state,accepting] (\New) [right=of \Old] {\New};

            \foreach \Source/\Target/\Label in {%
                1/2/nature,
                2/3/al,
                3/4/ize,
                4/5/ation}
                \path[->] (\Source) edge node [above] {\Label} (\Target);
            \path[->] (0) edge node[right] {de} (1);
        \end{tikzpicture}
    \end{center}
    
    But this automaton is too general, it also allows patterns like \emph{denature} or \emph{denatural}.
    This is a little tricker to fix: we add a few more states to keep track of the fact that after \emph{de}, the first final state can only be one reached by \emph{ize}.
    %
    \begin{center}
        \begin{tikzpicture}
            \node[state,initial] (0) {0};

            % upper part
            \foreach \New/\Old in {%
                1'/0,
                2'/1',
                3'/2'%
                }
                \node[state] (\New) [right=of \Old] {\New};

            % lower part
            \node[state,initial] (1) [below=of 0] {1};
            \foreach \New/\Old in {%
                2/1,
                3/2,
                4/3,
                5/4%
                }
                \node[state,accepting] (\New) [right=of \Old] {\New};

            \foreach \Source/\Target/\Label in {%
                1/2/nature,
                2/3/al,
                3/4/ize,
                4/5/ation,
                0/1'/de,
                1'/2'/nature,
                2'/3'/al%
                }
                \path[->] (\Source) edge node [above] {\Label} (\Target);
            \path[->] (3') edge node[right] {ize} (4);
        \end{tikzpicture}
    \end{center}
    %
    When you now try all possible paths from an initial state to a final state, you'll see that they are exactly
    %
    \begin{itemize}
        \item \emph{nature},
        \item \emph{natural},
        \item \emph{naturalize},
        \item \emph{denaturalize},
        \item \emph{naturalization},
        \item \emph{denaturalization}.
    \end{itemize}
\end{examplebox}

FSAs can be much more complex than this.
As long as the number of states is finite, pretty much any graph you can imagine is an FSA\@.
You can have multiple arcs leaving a node, multiple arcs leading to the same node, or arcs that lead back to a previous node, possibly even the one they came from (\emph{loops}).
That is to say, you can have convoluted automata like the one below:
%
\begin{center}
    \begin{tikzpicture}
        \node[state,initial] (0) {0};
        \node[state] (1) [right=of 0] {1};
        \node[state,accepting] (2) [above=4em of 1] {2};
        \node[state] (3) [right=8em of 1] {3};

        \path[->] (0) edge [bend left=65]  node [above] {John} (1)
                      edge                 node [above] {Mary} (1)
                      edge [bend right=65] node [below] {Bill} (1)
                  (1) edge                 node [right] {left} (2)
                      edge [bend right=20] node [below] {thinks} (3)
                  (3) edge [bend right=60] node [above] {John} (1)
                  (3) edge [bend right=20] node [above] {Mary} (1)
                  (3) edge [bend left=60]  node [below] {Bill} (1);
    \end{tikzpicture}
\end{center}

\medskip
\begin{homework}
    What collection of sentences is described by the automaton above?
\end{homework}

\section{Some curious limits of language}

Linguists put the parts that make up \emph{denaturalization} into three distinct groups:
%
\begin{description}
    \item[stem] the base form, which can appear by itself; \emph{nature}
    \item[prefix] a part that can only appear before a stem; \emph{de-}
    \item[suffix] a part that can only appear after a stem; \emph{-al}, \emph{-ize}, \emph{-ation}
\end{description}
%
\begin{homework}
    Prefixes and suffixes can sometimes be iterated.
    Your grandfather's father is your \emph{great grandfather}, whose father is your \emph{great great grandfather}, whose father is your \emph{great great great grandfather}, and so on.
    Try to write an FSA that produces \emph{grandfather}, \emph{great grandfather}, and so on.
    You may analyze \emph{grandfather} as a stem.
\end{homework}
%
\begin{homework}
    Of course we also have \emph{great grandmother}, \emph{great great grandmother}, and so on.
    How could the automaton from the previous example be modified to also allow these words?
\end{homework}

Those are not the only classes --- among other things there are also \emph{circumfixes}. 
An example of a circumfix is German \emph{ge- -t}, which is used to form the participle form of a verb.
So \emph{rauf-en} `brawl-to' can be turned into \emph{ge-rauf-t} `(has) brawled'.

\medskip
\begin{homework}
    German actually has a split with respect to this circumfix.
    For some verbs it is \emph{ge- -t}, for others it is \emph{ge- -en}.
    And for some verbs, the form is irregular.
    %
    \begin{center}
        \begin{tabular}{lll}
            \toprule
            \textbf{Verb stem}
            &
            \textbf{Infinitival form}
            &
            \textbf{Past participle form}\\
            \midrule
            lauf `run'& laufen & gelaufen\\
            rauf `brawl' & raufen & gerauft\\
            sauf `guzzle' & saufen & gesoffen\\
            tauf `baptize' & taufen & getauft\\
            \bottomrule
        \end{tabular}
    \end{center}
    %
    Write an FSA that produces all the correct infinitival and past participle forms, and only those.
    Try to keep the number of states as small as possible.
\end{homework}

One would expect that languages can freely choose whether they use prefixes, suffixes or circumfixes.
But this does not seem to be the case.
What we find is that across languages, circumfixes do not seem to be iterable.
%
\begin{examplebox}[Saying `the day after tomorrow' in German and Ilocano]
    German has a much simpler way than English to say `the day after tomorrow'.
    You just take \emph{morgen} (the word for `tomorrow') and add the prefix \emph{über} `over'.
    What makes this even nicer is that \emph{über} can be iterated.
    %
    \begin{enumerate}
        \item morgen `tomorrow'
        \item über-morgen `the day after tomorrow'
        \item über-über-morgen `the day after the day after tomorrow'
        \item \ldots
    \end{enumerate}
    %
    Ilocano (an Austronesian language spoken on the Philippines) also has a single word for `the day after tomorrow', but it is built with the circumfix \emph{ka- -an} instead of a prefix.
    %
    \begin{enumerate}
        \item bigat `tomorrow'
        \item ka-bigat-an `the day after tomorrow'
    \end{enumerate}
    %
    In contrast to German \emph{über-}, however, Ilocano \emph{ka- -an} cannot be freely iterated.
    So you cannot say something like \emph{ka-ka-bigat-an-an}.
\end{examplebox}
%
\begin{homework}
    It is very easy to give an FSA for the German pattern --- it's just a variation of the \emph{great $\cdots$ great grandfather} pattern we saw before for English.
    Similarly, an automaton for the Ilocano pattern is easy to come up with.
    But what if Ilocano were also unbounded, so that we could also have words like \emph{ka-ka-ka-bigat-an-an-an}, i.e.\ words with the same number of \emph{ka-} prefixes and \emph{-an} suffixes.
    Can you write an automaton for this?
\end{homework}
%
The difference between German \emph{über-} and Ilocano \emph{ka- -an} is an interesting puzzle, and it is part of a much larger observation: prefixes and suffixes can be freely iterated, but circumfixes cannot.
Why should languages work this way?
FSAs provide an answer.


\section{The limits of finite-state automata (aka the hard part)}

FSAs are fairly powerful devices, and they have found many applications in the real world.
They control elevators, help biologists with genome sequencing, give you word suggestions when you write a text message, and are even involved in the automatic creation of subtitles for Youtube videos.
Tons of technology uses FSAs, including language technology.

But FSAs cannot do everything.
There is a very strict bound on the complexity of the patterns they can handle.
And what we will see now is that even though FSAs can iterate prefixes and suffixes (as you've already shown yourself with the FSA for \emph{great $\cdots$ great grandfather}), they cannot correctly iterate circumfixes.

Before we state the theorem, let us think about how FSAs actually work.
In particular, what do the states stand for?
Suppose you have taken a path that leads to state $1$ in the convoluted automaton from the previous section.
How can you continue to get a well-formed pattern?
Well, by taking any path that leads you from $1$ to a final state.
There may be multiple paths to choose from, but anyone will do.
It does not matter how you actually got into $1$.
Whether you came from $0$ to $1$ via \emph{John}, \emph{Mary}, or \emph{Bill} is completely irrelevant, the only thing that matters is that you are in state $1$ now and may take any path that gets you from $1$ to $2$.
%
\begin{description}
    \item[Crucial Insight 1]
        If you have paths that take you from an initial state to the same state, then those paths can be continued in exactly the same fashion towards a final state.
\end{description}

FSAs have only finitely many states.
Suppose we have some pattern, and we compare all paths that could possibly arise in that pattern.
We put two paths in the same bin if they can be continued in exactly the same fashion.
Then the pattern can be handled by an FSA only if we end up with a finite number of distinct bins.
%
\begin{examplebox}{Bins for the \emph{great $\cdots$ great grandfather} Pattern}
    Consider the pattern \emph{grandfather}, \emph{great grandfather}, \emph{great great grandfather}, and so on.
    If the pattern starts with \emph{grandfather}, then there is nothing else that can be added, so there are no continuation paths at all.
    For \emph{great}, one possible continuation path is \emph{grandfather}, producing \emph{great grandfather}.
    But we could also continue with \emph{great grandfather}, \emph{great great grandfather}, and so on.
    Here's a table:
    %
    \begin{center}
        \begin{tabular}{rl}
            \toprule
            \textbf{Incoming Path} & \textbf{Continuation Paths}\\
            \midrule
            grandfather & ---\\
            great & grandfather, great grandfather, \ldots\\
            great great & grandfather, great grandfather, \ldots\\
            great great great & grandfather, great grandfather, \ldots\\
            $\vdots$
            \\
            \bottomrule
        \end{tabular}
    \end{center}
    %
    So there's only two bins for paths.
    One is for \emph{grandfather}, which cannot be continued.
    The other one is for any path that starts with \emph{great}, as they can all be continued by an arbitrary number of \emph{great}s followed by \emph{grandfather}.
    Since we only have two bins, the pattern can be handled by an FSA\@.
\end{examplebox}

The bins are essentially the states in the automaton: two strings are in the same bin because they take you to the same state in the automaton, and from there you can proceed in a specific manner irrespective of how you got there.
But a finite-state automaton can only have a finite number of states, so it can only describe patterns where the number of distinct bins is finite.

\begin{description}
    \item[Crucial Insight 2]
    A pattern can be captured by an FSA only if its paths can be grouped into finitely many bins.
\end{description}

This insight allows us to show that unbounded circumfixation is impossible with FSAs.
The proof is a joint class exercise.
%
% \begin{examplebox}{Bins for `tomorrow' Circumfixation in Ilocano}
%     Suppose that the circumfix in Ilocano could be freely iterated, as long as the number of \emph{ka-} matches the number of \emph{-an}.
%     Then we would have the following table:
%     %
%     \begin{center}
%         \begin{tabular}{rl}
%             \toprule
%             \textbf{Incoming Path} & \textbf{Continuation Paths}\\
%             \midrule
%             bigat & ---\\
%             \midrule
%             ka & bigat an, ka bigat an an, ka ka bigat an an an, \ldots\\
%             ka bigat & an\\
%             ka bigat an & ---\\
%             \midrule
%             ka ka &  bigat an an, ka bigat an an an, \ldots\\
%             ka ka bigat & an an\\
%             ka ka bigat an & an\\
%             ka ka bigat an an & ---\\
%             \midrule
%             ka ka ka & bigat an an an, ka bigat an an an an, \ldots\\
%             ka ka ka bigat & an an an\\
%             ka ka ka bigat an & an an\\
%             ka ka ka bigtan an an & an\\
%             $\vdots$
%             \\
%             \bottomrule
%         \end{tabular}
%     \end{center}
%     %
%     Some strings have the same continuation and thus end up in the same bin, e.g.~\emph{ka bigat} and \emph{ka ka bigat an}.
%     But it is also very obvious that the strings \emph{ka}, \emph{ka ka}, \emph{ka ka ka}, and so on, all have different continuations.
%     If the circumfix can be freely iterated, then there are infinitely many such strings, each one going in a different bin, so that there must be infinitely many bins.
%     But then the pattern would also require infinitely many states, so it cannot be done with an FSA\@.
% \end{examplebox}

\section{Wrapping up}

The iterability difference between prefixes and suffixes on the one hand and circumfixes on the other is really puzzling from non-mathematical perspective.
But once we look at them from a mathematical perspective, we see that there is a huge complexity difference: FSAs can iterate prefixes and suffixes, but not circumfixes.
So if the mechanisms our brain uses to handle word structure are similar to FSAs, then it is not surprising that circumfixes are never iterated in any languages --- the relevant mechanisms simply cannot do it!

We have accomplished something that few people would consider possible: we have studied language from a mathematical perspective, and we were able to explain properties that are shared by all human languages by purely mathematical means.

\bigskip
\fcolorbox{brown!25}{brown!25}{%
    \textbf{The Take Home Message.\quad}
    Mathematics explains language!
}

\bigskip
\begin{homework}
    Word structure indeed seems to be fully captured by FSAs.
    But there are ways to show that the structure of English sentences is more complicated then what FSAs can handle.
    Do you have an idea what the argument might be?
    
    \smallskip
    \textbf{Hint:} The sentences below play a crucial role.
    %
        \begin{itemize}
            \item A man left.
            \item A man that a woman saw left.
            \item A man that a woman that a woman saw saw left.
            \item A man that a woman that a woman that a woman saw saw saw left.
        \end{itemize}
\end{homework}
